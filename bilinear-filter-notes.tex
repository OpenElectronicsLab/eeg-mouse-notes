\documentclass[a4paper]{article}
% https://www.writelatex.com/1267010tfjbxp#/3080075/

\usepackage[english]{babel}
\usepackage[utf8]{inputenc}
\usepackage{amsmath}
\usepackage{graphicx}
\usepackage[colorinlistoftodos]{todonotes}
\usepackage[margin=3cm]{geometry}

\title{Bilinear filter math notes}

\author{Kendrick, Ace, Eric}

\date{\today}

\begin{document}
\maketitle

Following Numerical Recipes 3rd ed. by Press et al,

\begin{equation}
z = e^{-2\pi i f \Delta} = \frac{1+iw}{1-iw} \label{eq:z}
\end{equation}
or equivalently
\begin{equation}
w = i \left(\frac{1-z}{1+z}\right) \label{eq:w}
\end{equation}
Where $\Delta$ is our sampling interval.
This implies
\begin{equation}
w \equiv \tan(\pi f \Delta).
\end{equation}

We can describe a digital filter as
\begin{equation}
y_n = \sum^{M}_{k=0} c_k x_{n-k} + \sum^{N-1}_{j=0} d_j y_{n-j-1}.
\end{equation}
Expanding this equation for $N=2$, we have
\begin{equation}
y_n = c_0 x_n + c_1 x_{n-1} + c_2 x_{n-2} + d_0 y_{n-1} + d_1 y_{n-2}.\label{eq:expanded_coeficients}
\end{equation}

Knowing the coefficients $c_k$ and $d_j$ we can calculate the filter's response to a given frequency as
\begin{equation}
\mathcal{H}(f) = \frac{\sum^{M}_{k=0} c_k e^{-2\pi i k f \Delta}}{1-\sum^{N-1}_{j=0} d_j e^{-2\pi i (j+1) f \Delta}}.
\end{equation}

We can rewrite this in terms of $z$ as
\begin{equation}
\mathcal{H}(f) = \frac{\sum^{M}_{k=0} c_k z^{-k}}{1-\sum^{N-1}_{j=0} d_j z^{-(j+1)}},
\end{equation}
or for $N=M=2$,
\begin{equation}
\mathcal{H}(f) = \frac{c_0 + c_1 z^{-1} + c_2 z^{-2}}{1 - d_0 z^{-1} - d_1 z^{-2}}.\label{eq:freq_response}
\end{equation}

\section{2nd Order Bandpass Filter}
We can design the amplitude of a filter response function $\mathcal{H}$ for a band pass filter as
\begin{equation}
|\mathcal{H}(f)|^2 = \left(\frac{w^2}{w^2 + a^2}\right)\left(\frac{b^2}{w^2 + b^2}\right).
\end{equation}
where
\begin{equation}
a = \tan(\pi f_\textrm{high} \Delta)
\end{equation}
is our high pass cutoff, and
\begin{equation}
b = \tan(\pi f_\textrm{low} \Delta)
\end{equation}
is our low pass filter.
For example, we could choose
\begin{equation}
\mathcal{H}(f) = \left(\frac{w}{w - ia}\right)\left(\frac{ib}{w - ib}\right).\label{eq:filter_choice}
\end{equation}
We can re-write this as
\begin{align}
\mathcal{H}(f) & = \left(\frac{i \left(\frac{1-z}{1+z}\right)}{i \left(\frac{1-z}{1+z}\right) - ia}\right)\left(\frac{ib}{i \left(\frac{1-z}{1+z}\right) - ib}\right) \\
 & = \left(\frac{ \left(\frac{1-z}{1+z}\right)}{ \left(\frac{1-z}{1+z}\right) - a}\right)\left(\frac{b}{\left(\frac{1-z}{1+z}\right) - b}\right) \\
 & = \left(\frac{\left(1-z\right)}{ \left(1-z\right) - a(1+z)}\right)\left(\frac{b\left(1+z\right)}{\left(1-z\right) - b\left(1+z\right)}\right) \\
 & = \left(\frac{1-z}{ 1-z - a -az }\right)\left(\frac{b+bz}{1-z- b-bz}\right) \\
 & = \frac{b + bz -zb -bz^2}{1 - z - b - bz -z + z^2 +zb + bz^2 -a +za +ab + abz  -az + az^2 + azb + abz^2 } \\
 & = \frac{b-bz^2}{1 -b -a +ab -2z +2abz +az^2 + abz^2 +z^2 +bz^2} \\
 & = \frac{b-bz^2}{(1 -b  -a +ab) - (2 +2ab)z + (a + ab + 1 +b )z^2} \\
 & = \frac{b-bz^2}{(1-a)(1-b) - (2 +2ab)z + (1 + a)(1 + b)z^2} \\
 & = \frac{bz^{-2}-b}{(1-a)(1-b)z^{-2} - (2 +2ab)z^{-1} + (1 + a)(1 + b)} \\
 & = \frac{-\frac{b}{(1 + a)(1 + b)}+\frac{b}{(1 + a)(1 + b)}z^{-2}}{ \frac{(1-a)(1-b)}{(1 + a)(1 + b)}z^{-2} - \frac{2 +2ab}{(1 + a)(1 + b)}z^{-1} + 1} \\
  & = \frac{-\frac{b}{(1 + a)(1 + b)} + (0) z^{-1} + \frac{b}{(1 + a)(1 + b)}z^{-2}}{1  - \frac{2 +2ab}{(1 + a)(1 + b)}z^{-1} - \left(-\frac{(1-a)(1-b)}{(1 + a)(1 + b)}\right)z^{-2}}
\end{align}
Comparing this to (\ref{eq:freq_response}), we have
\begin{align}
c_0 &= -\frac{b}{(1 + a)(1 + b)} \label{eq:c0}\\
c_1 &= 0 \\
c_2 &= \frac{b}{(1 + a)(1 + b)} \\
d_0 &= \frac{2 +2ab}{(1 + a)(1 + b)} \\
d_1 &= -\frac{(1-a)(1-b)}{(1 + a)(1 + b)} \label{eq:d1}
\end{align}

\section{High Pass Filter of order $2 n$}
The high pass cutoff of the above bandpass filter is unfortunately not very
steep.  To make a more selective filter we can choose a squared
amplitude response with a steeper cut off, for example
\begin{align}
    |\mathcal{H}(f)|^2 &= \frac{w^{4n}}{w^{4n} + a^{4n}}.
\end{align}

We will next find a function $\mathcal{H}(f)$ that has this squared amplitude
and has poles (roots in the denominator) with positive imaginary components,
which is required for stability of the filter.
Note that the denominator will be zero when $w^{4n} = -a^{4n} = (-1)^{4n}
a^{4n}$, i.e. it is a polynomial whose roots are the product of each of the
$4n$ roots of $-1$ with the conventional (positive real) $4n$-th root of $a$.
Similarly, the numerator's roots are all $0$.  We can use these observations to
factor the amplitude response as follows:
\begin{align}
    |\mathcal{H}(f)|^2 &= \frac{w^{4n}}{w^{4n} + a^{4n}} \\
                       &=  \frac{\prod_{j=0}^{4 n - 1}(w - 0)}{\prod_{j=0}^{4 n - 1}\left(w - \sqrt[4n]{a}\exp\left(\frac{i \pi (1 + 2 j)}{4 n}\right)\right)} \\
                       &= \prod_{j=0}^{4 n - 1} \left(\frac{w}{w - \sqrt[4n]{a}\exp\left(\frac{i \pi (1 + 2 j)}{4 n}\right)}\right) \\
                       &= \prod_{j=0}^{2 n - 1} \left(\frac{w}{w - \sqrt[4n]{a}\exp\left(\frac{i \pi (1 + 2 j)}{4 n}\right)}\right)
                          \prod_{j=2n}^{4 n - 1} \left(\frac{w}{w - \sqrt[4n]{a}\exp\left(\frac{i \pi (1 + 2 j)}{4 n}\right)}\right) \\
                       &= \prod_{j=0}^{2 n - 1} \left(\frac{w}{w - \sqrt[4n]{a}\exp\left(\frac{i \pi (1 + 2 j)}{4 n}\right)}\right)
                          \prod_{k=0}^{2 n - 1} \left(\frac{w}{w - \sqrt[4n]{a}\exp\left(\frac{i \pi (1 + 2 (4 n - 1 - k)}{4 n}\right)}\right) \\
                       &= \prod_{j=0}^{2 n - 1} \left(\frac{w}{w - \sqrt[4n]{a}\exp\left(\frac{i \pi (1 + 2 j)}{4 n}\right)}\right)
                          \prod_{k=0}^{2 n - 1} \left(\frac{w}{w - \sqrt[4n]{a}\exp\left(\frac{i \pi (8n - 1 - 2 k)}{4 n}\right)}\right) \\
                       &= \prod_{j=0}^{2 n - 1} \left(\frac{w}{w - \sqrt[4n]{a}\exp\left(\frac{i \pi (1 + 2 j)}{4 n}\right)}\right)
                          \prod_{k=0}^{2 n - 1} \left(\frac{w}{w - \sqrt[4n]{a}\exp\left(\frac{-i \pi (1 + 2 k)}{4 n}\right)}\right) \\
                       &= \left(\prod_{j=0}^{2 n - 1} \left(\frac{w}{w - \sqrt[4n]{a}\exp\left(\frac{i \pi (1 + 2 j)}{4 n}\right)}\right)\right)
                          \left(\prod_{j=0}^{2 n - 1} \left(\frac{w}{w - \sqrt[4n]{a}\exp\left(\frac{i \pi (1 + 2 j)}{4 n}\right)}\right)\right)^{*}.
\end{align}
Note that the first term is a rational function of $w$ whose poles (roots of
the denominator) all lie above the real axis as required for stability.  The
second term is simply the complex conjugate of the first term.  We thus choose
\begin{align}
    \mathcal{H}(f) &= \prod_{j=0}^{2 n - 1} \left(\frac{w}{w - \sqrt[4n]{a}\exp\left(\frac{i \pi (1 + 2 j)}{4 n}\right)}\right).
\end{align}

We now factor this function into quadratic terms, as cascaded quadratic filters
can be more stable than a single higher order filter of the same total order.
To do this, we note the symmetry of the roots across the imaginary axis and
pair each root with its mirror.
\begin{align}
    \mathcal{H}(f) &= \prod_{j=0}^{2 n - 1} \left(\frac{w}{w - \sqrt[4n]{a}\exp\left(\frac{i \pi (1 + 2 j)}{4 n}\right)}\right) \\
        &= \prod_{j=0}^{n - 1} \left(\frac{w}{w - \sqrt[4n]{a}\exp\left(\frac{i \pi (1 + 2 j)}{4 n}\right)}\right)
           \prod_{j=n}^{2 n - 1} \left(\frac{w}{w - \sqrt[4n]{a}\exp\left(\frac{i \pi (1 + 2 j)}{4 n}\right)}\right) \\
        &= \prod_{j=0}^{n - 1} \left(\frac{w}{w - \sqrt[4n]{a}\exp\left(\frac{i \pi (1 + 2 j)}{4 n}\right)}\right)
           \prod_{k=0}^{n - 1} \left(\frac{w}{w - \sqrt[4n]{a}\exp\left(\frac{i \pi (1 + 2 (2 n - 1 - k)}{4 n}\right)}\right) \\
        &= \prod_{j=0}^{n - 1} \left(\frac{w}{w - \sqrt[4n]{a}\exp\left(\frac{i \pi (1 + 2 j)}{4 n}\right)}\right)
           \prod_{k=0}^{n - 1} \left(\frac{w}{w - \sqrt[4n]{a}\exp\left(i \pi-\frac{i \pi (1 + 2 k)}{4 n}\right)}\right) \\
        &= \prod_{j=0}^{n - 1} \left(\frac{w}{w - \sqrt[4n]{a}\exp\left(\frac{i \pi (1 + 2 j)}{4 n}\right)}\right)
           \prod_{k=0}^{n - 1} \left(\frac{w}{w + \sqrt[4n]{a}\exp\left(-\frac{i \pi (1 + 2 k)}{4 n}\right)}\right) \\
        &= \prod_{j=0}^{n - 1}
           \left(\frac{w}{w - \sqrt[4n]{a}\exp\left(\frac{i \pi (1 + 2 j)}{4 n}\right)}\right)
           \left(\frac{w}{w + \sqrt[4n]{a}\exp\left(-\frac{i \pi (1 + 2 j)}{4 n}\right)}\right) \\
        &= \prod_{j=0}^{n - 1}
           \frac{w^2}{w^2 - w \sqrt[4n]{a} \left(\exp\left(\frac{i \pi (1 + 2 j)}{4 n}\right) -
           \exp\left(-\frac{i \pi (1 + 2 j)}{4 n}\right)\right) - \sqrt[2n]{a}} \\
        &= \prod_{j=0}^{n - 1}
           \frac{w^2}{w^2 - 2 i w \sqrt[4n]{a} \sin\left(\frac{\pi (1 + 2 j)}{4 n}\right) - \sqrt[2n]{a}}.
\end{align}
We next rewrite this equation in terms of $z$, so that
\begin{align}
    \mathcal{H}(f) &= \prod_{j=0}^{n - 1}\frac{w^2}
        {w^2 - 2 i w \sqrt[4n]{a} \sin\left(\frac{\pi (1 + 2 j)}{4 n}\right) - \sqrt[2n]{a}} \\
    &= \prod_{j=0}^{n - 1}\frac{\left(i\left(\frac{1 - z}{1 + z}\right)\right)^2}
        {\left(i\left(\frac{1 - z}{1 + z}\right)\right)^2 - 2 i \left(i\left(\frac{1 - z}{1 + z}\right)\right) \sqrt[4n]{a} \sin\left(\frac{\pi (1 + 2 j)}{4 n}\right) - \sqrt[2n]{a}} \\
    &= \prod_{j=0}^{n - 1}\frac{-\left(\frac{1 - z}{1 + z}\right)^2}
        {-\left(\frac{1 - z}{1 + z}\right)^2 + 2 \left(\frac{1 - z}{1 + z}\right) \sqrt[4n]{a} \sin\left(\frac{\pi (1 + 2 j)}{4 n}\right) - \sqrt[2n]{a}} \\
    &= \prod_{j=0}^{n - 1}\frac{\left(\frac{1 - z}{1 + z}\right)^2}
        {\left(\frac{1 - z}{1 + z}\right)^2 - 2 \left(\frac{1 - z}{1 + z}\right) \sqrt[4n]{a} \sin\left(\frac{\pi (1 + 2 j)}{4 n}\right) + \sqrt[2n]{a}} \\
    &= \prod_{j=0}^{n - 1}\frac{(1 - z)^2}
        {(1 - z)^2 - 2 (1 - z)(1 + z) \sqrt[4n]{a} \sin\left(\frac{\pi (1 + 2 j)}{4 n}\right) + (1 + z)^2 \sqrt[2n]{a}}.
\end{align}
For convenience, we let $\alpha = \sqrt[2n]{a}$ and $\gamma_j = 2 \sqrt[4n]{a} \sin\left(\frac{\pi (1 + 2 j)}{4 n}\right)$.
Continuing with these new variables, we have
\begin{align}
    \mathcal{H}(f) &= \prod_{j=0}^{n - 1}\frac{(1 - z)^2}
        {(1 - z)^2 - \gamma_j (1 - z)(1 + z) + (1 + z)^2 \alpha} \\
    &= \prod_{j=0}^{n - 1}\frac{1 - 2z + z^2}
        {1 - 2z + z^2 - \gamma_j + \gamma_j z^2 + \alpha + 2z \alpha + z^2 \alpha} \\
    &= \prod_{j=0}^{n - 1}\frac{z^2 - 2z + 1}
        {(1 + \alpha + \gamma_j)z^2 + 2(\alpha - 1) z + (1 + \alpha - \gamma_j)} \\
        &= \prod_{j=0}^{n - 1}\frac{1 - 2z^{-1} + z^{-2}}
        {(1 + \alpha + \gamma_j) - 2(1 - \alpha) z^{-1} - (\gamma_j - \alpha - 1)z^{-2}}.
\end{align}
Comparing this to (\ref{eq:freq_response}), we have
\begin{align}
    c_{j,0} &= \frac{1}{1 + \alpha + \gamma_j} \\
    c_{j,1} &= -\frac{2}{1 + \alpha + \gamma_j} \\
    c_{j,2} &= \frac{1}{1 + \alpha + \gamma_j} \\
    d_{j,0} &= \frac{2-2\alpha}{1 + \alpha + \gamma_j} \\
    d_{j,1} &= \frac{\gamma_j - \alpha - 1}{1 + \alpha + \gamma_j}.
\end{align}


\section{Stability}
Per Press et all, an IIR filter is only stable if all roots of
\begin{equation}
z^N - \sum^{N-1}_{j=0} d_j z^{(N-1)-j} = 0,\label{eq:stability}
\end{equation}
or in case with $N=2$
\begin{equation}
z^2 - (d_0 z + d_1) = 0,
\end{equation}
or equivalently
\begin{equation}
z^2 - d_0 z - d_1 = 0,
\end{equation}
lie withing the unit circle.

\section{todo}
\begin{itemize}
\item Restate (\ref{eq:expanded_coeficients}) in terms of (\ref{eq:c0})-(\ref{eq:d1})
\item Expand steps between (\ref{eq:z})-(\ref{eq:w})
\item Make choice of (\ref{eq:filter_choice}) less mysterious based on stability.
\item Explain stability criterion (\ref{eq:stability})
\end{itemize}
\end{document}
